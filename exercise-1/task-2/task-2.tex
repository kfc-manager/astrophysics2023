\section*{2. Artificial satellites on circular orbits}

a) The Hubble Space Telescope needs 95 minutes for one complete revolution arount the Earth. Why is it's
orbit not stable over several decades?\\
\\
The Hubble Space Telescope orbits only about 550 km above earth. At this low altitude there is still a 
tenuous atmosphere. Meaning that there are still thin layers of gas paricles, which drag a little on
the spacecraft. Over time that adds up and over several decades the orbit won't be stable. Besides that
the Earth's mass is not equally distributed, which has an impact on the satellites orbit at that low
altitude.\\
\\
b) Re-derive Equation (4.15) in the lecture notes.\\
\\
c) Suppose that a spacecraft moves on a circular orbit around the Earth at a height of $H = 1000$ km
above the surface. Now the spacecraft fires its rockets in tangential direction, accelerating its speed
to 1.2x its equilibrium orbital velocity. Calculate the new equilibrium orbit that will result from this
action (always assuming circular motion). At what height $H'$ above Earth will the spacecraft end up?\\
\\
The initial angular momentum of the spacecraft would be:
\begin{equation*}
    L_{initial} = m \cdot v_{initial} \cdot r_{initial}
\end{equation*}
where $r_{initial}$ would be the radius of the initial orbit. The final angular momentum would then be:
\begin{equation*}
    L_{final} = m \cdot v_{final} \cdot r_{final}
\end{equation*}
Because the law of conservation of angular momentum we can observe the following:
\begin{equation*}
    L_{final} = m \cdot v_{final} \cdot r_{final}
\end{equation*}
d) What would happen if the spacecraft in c) was instead accelerated to 1.5x its equilibrium orbital
velocity?\\
\\
